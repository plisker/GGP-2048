\documentclass[12pt]{article}


\usepackage[margin=1in]{geometry}
\usepackage{latexsym}
\usepackage{amssymb}
\usepackage[pdftex]{graphicx}
\usepackage{tikz}
\usepackage{mathtools}
\usepackage{xfrac}
\usepackage{hyperref}
\date{}

\usepackage{setspace}
\doublespacing

\def\B{{\mathbb B}}
\def\C{{\mathbb C}}
\def\D{{\mathbb D}}
\def\H{{\mathbb H}}
\def\M{{\mathbb M}}
\def\N{{\mathbb N}}
\def\P{{{\mathbb P}}}
\def\Q{{\mathbb Q}}
\def\R{{\mathbb R}}
\def\T{{\mathbb T}}
\def\Z{{\mathbb Z}}
\def\S{{\mathcal{S}}}
\def\s{{\vec{s}}}

\hypersetup{
    bookmarks=true,         % show bookmarks bar?
    unicode=false,          % non-Latin characters in Acrobat�s bookmarks
    pdftoolbar=true,        % show Acrobat�s toolbar?
    pdfmenubar=true,        % show Acrobat�s menu?
    pdffitwindow=false,     % window fit to page when opened
    pdfstartview={FitH},    % fits the width of the page to the window
    pdftitle={My title},    % title
    pdfauthor={Author},     % author
    pdfsubject={Subject},   % subject of the document
    pdfcreator={Creator},   % creator of the document
    pdfproducer={Producer}, % producer of the document
    pdfkeywords={keyword1, key2, key3}, % list of keywords
    pdfnewwindow=true,      % links in new PDF window
    colorlinks=true,       % false: boxed links; true: colored links
    linkcolor=red,          % color of internal links (change box color with linkbordercolor)
    citecolor=green,        % color of links to bibliography
    filecolor=magenta,      % color of file links
    urlcolor=cyan           % color of external links
}


%	\includegraphics[width=\textwidth]{slide1}
\makeatletter
\def\@seccntformat#1{%
  \expandafter\ifx\csname c@#1\endcsname\c@section\else
  \csname the#1\endcsname\quad
  \fi}
\makeatother
\renewcommand{\thesubsection}{}


\begin{document}

\title{CS 182: Artificial Intelligence\\Project Proposal}
\date{Fall 2016}
\author{Ahmed Ahmed \& Paul Lisker}

\maketitle

\renewcommand{\abstractname}{Topic Proposal}
\begin{abstract}
The purpose of this project will be to create a General Game Playing agent for the domain of games similar to and including \href{https://gabrielecirulli.github.io/2048/}{2048}. In other words, this agent should be able to successfully play a the game upon being presented with only the basic rules.
\end{abstract}

\subsection{Domain of Games} Since this project will concentrate on general game playing on the domain of 2048-like games, it is worth beginning by a description of this domain. The original 2048 game challenges the player to slide tiles around, merging with other like-tiles to create tiles of higher value until the desired 2048 tile is created, at which point the player wins the game. The player loses if at any point the board has become completely crowded (with every move, a tile with value 2 or 4 is randomly generated at any empty space) and no legal moves are available. While the goal of the game is to reach the 2048 tile, the original game also keeps track of a scoreboard; each tile merge increases the score by the value of the new tile.

Since 2048 is a fairly unique game, we intend to create several variations of the game to create a more robust domain of game. Our proposed variations include, but are not restricted to:
\begin{itemize}
\item Changing the gamespace from a $4\times4$ grid to an $n\times m$ grid, with $n,m \in \{3,4,5\}$
\item Benchmark $2^k$ tile for a non-zero score, with $k\in\{1,\dots\}$. In other words, if this minimum tile is not reached, the total score is zero.
\item Score evaluation functions. Two main categories, and accompanying examples:
\begin{itemize}
\item Cumulative score
\begin{itemize}
\item When two tiles are merged, add score of resulting tile
\end{itemize}
\item Final state score
\begin{itemize}
\item Sum tiles on final game space
\item Sum of tiles that have value $2^k$, where $k\bmod2=0$, minus the sum of tiles that have value $2^r$, where $r\bmod2=1$.
\end{itemize}

\end{itemize}

\end{itemize}



\paragraph{Assignment:} To ensure that you choose an appropriate project, you are required to turn in a 1�2 page project proposal. The proposal should begin with a clear, unambiguous statement of your topic, and include all of the following:
\begin{enumerate}
\item a brief discussion of the problem and algorithms you intend to investigate and the system you intend to build in doing so,
\item identification of specific related course topics (e.g. heuristic search, MDPs, CSPs, etc.).
\item examples of expected behavior of the system or the types of problems the algorithms you investigate are intended to handle,
\item the issues you expect to focus on,
\item and a list of papers or other resources you intend to use inform your project effort. This list will form the core of your project report reference list. If your project includes anything unusual (such as having significant systems demands), please state this as well.
\end{enumerate}

\end{document}